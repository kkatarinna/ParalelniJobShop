\documentclass[a4paper,11pt]{article}

\usepackage[utf8]{inputenc}
\usepackage[T2A]{fontenc}
\usepackage[serbian]{babel}
\usepackage{xcolor}
\usepackage{minted}
\usepackage{fancyvrb}

\VerbatimFootnotes

\setminted[python]{
    mathescape,
    linenos,
    numbersep=4pt,
    gobble=5,
    frame=lines,
    framesep=2mm,
    fontsize=\footnotesize
}

\setminted[cpp]{
    mathescape,
    linenos,
    numbersep=4pt,
    gobble=5,
    frame=lines,
    framesep=2mm,
    fontsize=\footnotesize
}

\usemintedstyle{pastie}

\author{Mladen Samardžić}
\title{Analiza jedne implementacije serijskog i paralelnog merge sort algoritma u programskom jeziku C++}
\date{}

\begin{document}
    \maketitle
    \thispagestyle{empty}
    \newpage

    \thispagestyle{empty}
    \mbox{}
    \newpage

    \tableofcontents
    \newpage

    \thispagestyle{empty}
    \mbox{}
    \newpage

    \section{Uvod}

    Sa naglim porastom paralelizacije računarskih sistema u proteklih nekoliko
    decenija, poraslo je i interesovanje za istraživanje i prikupljanje 
    algoritama koji bi takve vrste arhitektura iskoristili na što efikasniji
    način. Sortiranje kolekcija podataka jeedna je od osnovnih operacija, kao i
    deo mnogih složenijih, u bilo kom dovoljno kompleksnom sistemu; zato je od
    ključne važnosti odabir adekvatnog algoritma u odnosu na konkretne potrebe i optimizacija
    njegove implementacije u odnosu na sam sistem. Zbog svoje stabilnosti, merge
    sort pogodan je za sortiranje podataka po više kriterijuma (na primer, sortiranje
    osoba po prezimenu, pa imenu), a zbog svog divide-and-conquer pristupa problemu
    pogodan je za paralelizaciju. \par
    U ovom tekstu analiziraćemo performanse jedne implementacije ovog algoritma
    u programskom jeziku C++, poredeći ih sa performansama
    serijske implementacije istog algoritma i nove, paralelne verzije 
    \verb|std::stable_sort|, dostupne u standardnoj biblioteci od C++17 verzije jezika.
    \footnote{
        Iako je C++17 standardizovan 2017. godine, paralelne verzije STL algoritama
        za vreme pisanja ovog dokumenta implementirane su, što se tiče najpopularnijih kompajlera,
        samo u standardnim bibliotekama koje dolaze uz \verb|g++| 9+ i \verb|msvc++| 19.14+.
}
    Pri paralelizaciji algoritma koristićemo apstrakciju zadataka (\textit{tasks}), obezbeđenu od
    strane Intel-ove Thread Building Blocks biblioteke. Sva merenja vršićemo
    na Intel i7-6700HQ četvorojezgarnom (sa osam logičkih jezgara) procesoru i Ubuntu 19.04 operativnom
    sistemu, koristeći verziju 9.0.1 \verb|g++| kompajlera. Dodatne relevantne konfiguracije
    kompajlera navešćemo su u adekvatnim sekcijama.

    \newpage

    \section{Osnovne karakteristike serijskog merge sort-a}

    Osnovna ideja merge sort algoritma je jednostavna i može se izvesti principom
    matematičke indukcije na sledeći način: znamo da sortiramo prazan niz i niz sa jednim elementom --
    svaki takav niz je sortiran. Pretpostavimo onda da znamo da sortiramo nizove dužina
    $2,3, \dots, n$. Niz dužine $n + 1$ sortiraćemo tako što ga podelimo na dva podniza
    dužina $\lfloor \frac{n + 1}{2} \rfloor$ i $\lceil \frac{n + 1}{2} \rceil$, sortiramo
    te podnizove, pa ih prespojimo nazad u početni niz. Jednostavna Python implementacija
    izgledala bi ovako:
    \begin{minted}{python}
        def merge_sort(xs, start, end):
          if end - start < 2:
            return
          mid = (start + end) // 2
          merge_sort(xs, start, mid)
          merge_sort(xs, mid, end)
          merge(xs, start, mid, end)
    \end{minted}
    Takvu funkciju direktno bismo pozvali na sledeći način:
    \begin{minted}{python}
        merge_sort(xs, 0, len(xs))
    \end{minted}
    Ostaje nam, naravno, da definišemo proces spajanja sortiranih podnizova, što je tačno
    zadatak \verb|merge| funkcije u prethodnom primeru. Podnizove ćemo spajati tako
    što ćemo u isto vreme šetati po oba i porediti odgovarajuće elemente. Manji element
    dodaćemo u originalni niz, a indeks u podniz koji odgovara tom elementu povećaćemo za jedan,
    pošto smo se tog elementa rešili. Kada u potpunosti prođemo kroz jedan od podnizova, sve elemente
    drugog dodaćemo na kraj originalnog niza, jer su po preduslovima algoritma sortirani, a po
    čuvanim invarijantama veći od svih elemenata koje smo prethodno dodali. Vrativši se na
    primer, Python implementacija \verb|merge| izgledala bi ovako:
    \begin{minted}{python}
        def merge(xs, start, mid, end):
          left    = xs[start:mid]
          right   = xs[mid:end]
          i, j, k = 0, 0, 0
          while i < len(left) and j < len(right):
            if left[i] <= right[j]:
              xs[k] = left[i]
              i += 1
              k += 1
            else:
              xs[k] = right[j]
              j += 1
              k += 1
          while i < len(left):
            xs[k] = left[i]
            i += 1
            k += 1
          while j < len(right):
            xs[k] = right[j]
            j += 1
            k += 1
    \end{minted}
    Primetimo nekoliko stvari:
    \begin{itemize}
        \begin{item}
            Morali smo da izdvojimo dodatnih $O(\verb|end| - \verb|start|)$ memorije za podnizove \verb|left| i \verb|right|.
            To je nužno jer bismo inače izgubili neke elemente pri prepisivanju u originalni niz.
        \end{item}
        \begin{item}
            Pošto prolazimo kroz svaki element niza \verb|xs| tačno jednom (kroz kopije njegovih podnizova),
            vremenska kompleksnost je takođe $O(\verb|end| - \verb|start|)$.
        \end{item}
        \begin{item}
            Uslov na liniji 6 mogli smo napisati i kao \verb|left[i] < right[j]|, ali bismo
            onda izgubili jednu od najbitnijih osobina merge sort algoritma: njegovu stabilnost. Algoritam
            za sortiranje je stabilan ako čuva redosled elemenata koji se u svrhe sortiranja porede kao
            jednaki. Ovo nema nikakav značaj ukoliko radimo sa tipovima koji se mogu porediti samo na jedan
            način; ako, međutim, sortiramo radnike po prezimenu, možda nam je bitno da se radnicima sa istim prezimenima
            održi redosled u nizu (jer smo ih pre toga sortirali po odeljenju u kom rade).
        \end{item}
    \end{itemize}
    Kako sada imamo potpuni opis algoritma, možemo analizirati njegove asimptotske performanse.\par
    Vodeći se Python implementacijom, vreme izvršavanja možemo matematički opisati kao 
    $T(n) = 2T(\frac{n}{2}) + O(n)$. Jednostavnom primenom master teoreme na prethodnu
    diferentnu jednačinu lako se dolazi do zaključka da je vremenska kompleksnost merge sort-a $O(n \log n)$.\par
    Prostorna kompleksnost u potpunosti proističe iz faze spajanja podnizova, a, kako tada alociramo
    količinu memorije proporcionalnu dužini niza, potrebna memorija linearno je zavisna od broja elemenata niza
    (tj. prostorna kompleksnost je $O(n)$).\par
    \newpage
    
    \section{Paralelni merge sort}
    Vratimo se na prvi primer:
    \begin{minted}{python}
        def merge_sort(xs, start, end):
          if end - start < 2:
            return
          mid = (start + end) // 2
          merge_sort(xs, start, mid)
          merge_sort(xs, mid, end)
          merge(xs, start, mid, end)
    \end{minted}
    Dva rekurzivna poziva na linijama 5 i 6 međusobno su nezavisna, jer svaki od njih radi sa disjunktnim
    polovinama niza, pa u principu mogu da se izvršavaju paralelno. Trivijalnom transformacijom u pseudo-Python
    kod dobijamo prvu verziju paralelnog merge sort algoritma:
    \begin{minted}{python}
        def merge_sort_par_v1(xs, start, end):
          if end - start < 2:
            return
          mid = (start + end) // 2
          fork merge_sort_par_v1(xs, start, mid)
          merge_sort_par_v1(xs, mid, end)
          join
          merge(xs, start, mid, end)
    \end{minted}
    Pošto \verb|merge| očekuje potpuno sortirane podnizove na ulazu, moramo da sačekamo da se oba
    podniza sortiraju (tome služi \verb|join| na sedmoj liniji) pre nego što je pozovemo.
    Ovo rešenje, iako korak u pravom smeru, nije najoptimalnije u pogledu paralelizacije.\par
    Prvo, apstrakcije za rad sa konkurentnim i paralalenim kodom (nezavisno da li su to niti,
    zadaci, procesi, kanali ili nešto sasvim drugo) dolaze sa određenim dodatnim zahtevima po pitanju resursa.
    Na primer, za niz od 4 elementa, pozivanje funkcije iz prethodnog primera stvara 3 niti, što oduzima
    mnogo više vremena od samog sortiranja. Generalno, stvara se onoliko niti koliko ima levih grana
    u binarnom stablu poziva \verb|merge_sort_par_v1|.\par 
    Ovaj problem rešićemo tako što ćemo definisati konstantu \verb|cutoff|, koja će kontrolisati kada
    će se pozivati paralelna, a kada serijska verzija funkcije. Najpogodnija vrednost \verb|cutoff|
    suštinski nalazi se eskperimentalnim metodama i predstavlja balans između povećanja performansi
    deljenjem posla na više procesorskih jezgara i gubitka stvaranjem previše niti. Definisanje ovakve
    konstante nije specifično za merge sort, nego je karakteristika većine paralelnih algoritama baziranih
    na rekurziji. Sa uvedenom konstantom, naš pseudokod izgleda ovako:
    \begin{minted}{python}
        # cutoff defined somewhere else
        def merge_sort_par_v2(xs, start, end):
          if end - start <= cutoff:
            merge_sort(xs, start, end)
          mid = (start + end) // 2
          fork merge_sort_par_v2(xs, start, mid)
          merge_sort_par_v2(xs, mid, end)
          join
          merge(xs, start, mid, end)
    \end{minted}
    Ova promena trebalo bi da značajno poboljša performanse i za manje i za veće veličine nizova, u zavisnosti
    od izabrane \verb|cutoff| vrednosti.\par
    Druga tačka poboljšanja je paralelizacija samog procesa spajanja podnizova. Iz originalnog \verb|merge|
    koda nije očigledno da je moguće uraditi to na smislen način, ali jednostavnom promenom interfejsa
    i inherentno sekvencijalnog pristupa spajanju možemo doći do razumne verzije algoritma na relativno
    jednostavan način.\par
    Prvo, umesto da \verb|merge| spaja podnizove u mestu, spajaće ih u novi izlazni niz. Primetimo da ovo
    ne zahteva dodatnu memoriju: samo smo pomerili dužnost alociranja memorije izvan funkcije. Sada
    merge algoritam prima dva podniza (u daljem tesktu: levi i desni) kao ulazne parametre i niz u koji
    će se spajanje izvršiti kao izlazni parametar.\par
    Zatim, pretpostavićemo, bez gubljenja opštosti, da je dužina levog podniza veća ili jednaka dužini desnog.
    Ukoliko to nije slučaj, samo ćemo pozvati algoritam sa desnim podnizom kao levim i levim kao desnim.
    Uvešćemo sledeću notaciju: $p_k , 0 \le k < n$ su indeksi u levi podniz, $q_k , 0 \le k < m$ indeksi
    u desni podniz, a $p_n$ i $q_m$ indeksi koji označavaju krajeve odgovarajućih podnizova.
     Paralelni merge algoritam onda se može opisati na sledeći način:
    \begin{itemize}
        \begin{item}
            Izabraćemo srednji element levog podniza, $\bar{a}$. Neka je indeks tog elementa $r_1$. Onda je
            $r_1 = \lfloor \frac{p_1 + p_n}{2} \rfloor$.
        \end{item}
        \begin{item}
            Neka je $r_2$ indeks prvog elementa u desnom podnizu takvog da nije manji od $\bar{a}$.
        \end{item}
        \begin{item}
            Neka je $r_3$ indeks u izlazni niz gde će $\bar{a}$ biti smešten.
            Pošto znamo da elemenata manjih (ili jednakih) od $\bar{a}$ u levom podnizu ima $r_1$, a
            u desnom (striktno manjih) $r_2$, onda je $r_3$ = $r_1 - p_0$ + $r_2 - q_0$.
        \end{item}
        \begin{item}
            U izlazni niz na poziciju $r_3$ upisaćemo $\bar{a}$.
        \end{item}
        \begin{item}
            Pozvaćemo algoritam rekurzivno dva puta. Argumenti prvog poziva biće 
            levi podniz u intervalu indeksa $[p_0 , r_1)$, desni u intervalu
            indeksa $[q_0 , r_2)$ i izlazni niz u intervalu $[0, r_3)$. Argumenti
            drugog poziva biće levi podniz u intervalu indeksa $[r_1 + 1, p_n)$, desni u intervalu
            $[r_2, q_m)$ i izlazni niz u intervalu $[r_3 + 1, m + n)$. Ova dva poziva
            radiće nad disjunktnim podacima, pa se bezbedno mogu izvršavati iz različitih
            niti.
        \end{item}
    \end{itemize}
    U pseudo-Python-u, implementacija ovog algoritma imala bi sledeći oblik:
    \begin{minted}{python}
        # merge_cutoff defined somewhere else
        def merge_par(xs, p0, pn, q0, qm, out, oi):
          l_size = pn - p1
          r_size = qn - q1
          if l_size < r_size:
            l_size, r_size = r_size, l_size
            p0, q0 = q0, p0
            pn, qm = qm, pn
          if l_size == 0:
            return
          if l_size + r_size <= merge_cutoff:
            merge_into(xs, p0, pn, q0, qm, out, oi)
          r1 = (p1 + pn) // 2
          r2 = lower_bound(xs, q1, qm, xs[r1])
          r3 = oi + (r1 - p0) + (r2 - q0)
          out[r3] = xs[r1]
          fork merge_par(xs, p0, r1, q0, r2, out, oi)
          merge_par(xs, r1 + 1, pn, r2, qm, out, r3)
          join
    \end{minted}
    Ova implementacija dodaje parametar \verb|oi|, koji je suštinski potreban samo da bismo
    znali odakle treba indeksirati izlazni niz \verb|out|. Takođe, pretpostavićemo da su funkcije
    \verb|merge_into| i \verb|lower_bound| već implementirane; \verb|merge_into| je jednostavna
    adaptacija \verb|merge| funkcije od ranije interfejsu \verb|merge_par|
    (uz minimalnu promennu semantike tako da spaja u izlazni niz umesto u mestu),
    dok je \verb|lower_bound| efektivno binarna pretraga. Konačna verzija paralelnog merge sort-a
    onda izgleda ovako:
    \begin{minted}{python}
        # cutoff defined somewhere else
        def merge_sort_par_v3(xs, start, end):
          if end - start <= cutoff:
            merge_sort(xs, start, end)
          mid = (start + end) // 2
          fork merge_sort_par_v3(xs, start, mid)
          merge_sort_par_v3(xs, mid, end)
          join
          out = [0] * (end - start)
          merge_par(xs, start, mid, mid, end, out, 0)
    \end{minted}
    \newpage

    \section{Implementacija}
    Naše implementacije serijskog i paralelnog merge sort algoritma imaju cilj da
    budu generičke prirode: po tipu kontejnera koji se sortira, po tipu elemenata
    tog kontejnera i po predikatu koji se koristi radi odlučivanja da li je jedan
    element manji od drugog. Iz tog razloga, kao i zbog konzistentnosti sa već postojećim
    alatima pruženim od strane standardne biblioteke, interfejs naše implementacije
    baziran je na interfejsu \verb|std::sort|.\par
    U ovoj sekciji izdvojićemo anotirane najbitnije delove implementacije; zbog toga su
    većinom izostavljene \verb|#include| direktive i \verb|std::| i \verb|tbb::|
    prefiksi. Serijske verzije algoritama nalaze se u prostoru imena \verb|serial|,
    paralelne u \verb|parallel|, a nekoliko alijasa za tipove iteratora u \verb|common|
    prostoru imena.

    \subsection{Serijski merge i merge sort}
    Serijski merge sort direktan je prevod iz Python primera u generički C++:
    \begin{minted}{cpp}
        // include/serial/sort.hpp
        template <typename It, typename P = less<>>
        void serial::sort(It first, It last, P&& p = {}) {
          auto const size = static_cast<size_t>(last - first);
          if (size < 2)
            return;

          auto const mid = first + size/2;
          serial::sort(first, mid, p);
          serial::sort(mid, last, p);
          serial::merge(first, mid, last, p);
        }
    \end{minted}
    Sve funckije (tj. šabloni za funkcije) ove implementacije očekuju da 
    \verb|It| zadovoljava uslov \verb|LegacyRandomAccessIterator|, dok \verb|P| treba da bude
    callable (bilo pokazivač na funkciju, funktor ili lambda) koji prima dva
    argumenta tipova implicitno konvertibilnih u tip elementa kontejnera čiji je iterator
    \verb|It|, sa povratnim tipom implicitno konvertibilnim u \verb|bool|. Tip elemenata 
    kontejnera mora da bude \verb|MoveAssignable| i \verb|MoveConstructible|.\par
    \verb|merge| je podeljen u dve funkcije: \verb|merge| i \verb|merge_impl|:
    \begin{minted}{cpp}
      // include/serial/merge.hpp
      template <typename It, typename Out, typename P = less<>>
      void serial::merge_impl(It lf, It ll, It rf, It rl, Out of, P&& p) {
        while (lf != ll && rf != rl)
          if (!p(*rf, *lf))
            *of++ = move(*lf++);
          else
            *of++ = move(*rf++);
        of = move(lf, ll, of);
        move(rf, rl, of);
      }
    \end{minted}
    \verb|merge_impl| spaja dva intervala $[\verb|lf|, \verb|ll|)$ i $[\verb|rf|, \verb|rl|)$
    u kolekciju na koju pokazuje iterator \verb|out|. \verb|Out| treba da zadovoljava uslov
    \verb|LegacyOutputIterator|. U suštini, kao i \verb|sort| malopre, i ovo je skoro pa direktan prevod Python funkcije
    definisane ranije, s tim da utilizuje move semantiku specifičnu za C++ i umesto operatora $<$ koristi
    komparator \verb|P|, gde \verb|p(x, y)| vraća \verb|true| ukoliko je \verb|x| manje od \verb|y| i
    \verb|false| inače. Zbog toga je uslov \verb|!p(*rf, *lf)| ekvivalentan uslovu \verb|left[i] <= right[j]|
    iz Python implementacije, sa ulogom održavanja stabilnosti algoritma. Ova funkcija u suštini ne predstavlja
    javni interfejs (implicirano \verb|_impl| sufiksom), već se koristi pri implementaciji \verb|merge| i paralelnog
    merge algoritma (doduše, ništa osim proizvoljnog izbora od strane autora ne sprečava interfejs da bude javan, ako
    uzmemo u obzir da \verb|std::merge| semantički izgleda ekvivalentno).\par
    \verb|merge| se onda svodi na alociranje kontejnera u koji će se podnizovi spajaju:
    \begin{minted}{cpp}
        // include/serial/merge.hpp
        template <typename It, typename P = less<>>
        void serial::merge(It first, It mid, It last, P&& p = {}) {
          auto const size = static_cast<size_t>(last - first);
          auto alloc = allocator<value_t<It>>{};
          auto *into = alloc.allocate(size); 
          merge_impl(first, mid, mid, last, into, p);
          move(into, into + size, first);
          alloc.deallocate(into, size);
        }
    \end{minted}
    Ovde \verb|value_t<It>| predstavlja tip na koji pokazuje iterator tipa \verb|It|, a njegova se definicija
    nalazi u prostoru imena \verb|common|.
    Jedan detalj koji će pažljivom čitaocu zapasti za oko je taj da za dinamičku alokaciju niza u koji ćemo spajati podnizove umesto \verb|std::vector| ili operatora 
    \verb|new| koristimo \verb|std::allocator| i njegove funkcije članice \verb|allocate| i
    \verb|deallocate|. To smo uradili zato što i \verb|std::vector| i operator \verb|new|
    pozivaju konstruktore odgovarajućeg tipa, dok \verb|allocate| jednostavno alocira dovoljno
    prostora za taj tip, bez konstrukcije. Generalno, za proizvoljan tip \verb|T| ne možemo da garantujemo da
    podržava bilo koji određeni konstruktor. Najzad, u zavisnosti od tipa konstrukcija može biti skupa, a u sklopu
    samog algoritma nema nikakvu ulogu: bitno nam je da imamo dovoljno prostora u memoriji gde bismo mogli da spojimo
    podnizove, a ne šta se u toj memoriji nalazi pre samog procesa.
    
    \subsection{Paralelni merge i merge sort}

    Za realizaciju paralelnih algoritama koristićemo Intel-ovu TBB biblioteku i apstrakciju zadataka koju
    ona pruža. Svaki od naših algoritama biće umotan u klasu koja nasleđuje \verb|tbb::task| (ili neku njenu subklasu).
    Za \verb|tbb::task| specifična je funkcija članica \verb|execute|, čije telo predstavlja posao delegiran
    tom zadatku, a koja vraća pokazivač na sledeći zadatak koji treba izvršiti. Kompozicijom zadataka stvaramo
    usmeren aciklični graf (DAG) koji predstavlja ukupan algoritam koji implementiramo. U našem slučaju, DAG će
    biti jednostavno binarno stablo, ali dosta paralelnih algoritama zahteva komplikovaniju reprezentaciju
    i koordinaciju između individualnih zadataka.
    Kako poziv ovako realizovanog algoritma nije trivijalan, njega ćemo umotati u funkciju sa identičnim interfejsom
    onom u odgovarajućoj serijskoj verziji.\par
    Na dalje, posmatraćemo samo \verb|execute| funkcije klasa, jer se isključivo u njima zapravo dešava nešto. Sve
    promenljive članice direktno odgovaraju ulazim parametrima \verb|serial::sort| i \verb|serial::merge_impl|, a svi
    konstruktori samo postavljaju odgovarajuće promenljive u zadato stanje.\par
    Posmatraćemo prvo klasu \verb|sort_impl|:
    \begin{minted}{cpp}
        template <typename It, typename P>
        class parallel::sort_impl : public task {
        private:
          // member variables and sort_continuation definition
        public:
          // constructor
          // ...
          auto execute() -> task* {
            auto const size = static_cast<size_t>(last - first);
            if (size <= sort_cutoff) {
              serial::sort(first, last, p);
              return nullptr;
            }

            auto const mid = first + size/2;

            auto& c = *new (allocate_continuation()) sort_continuation{
              first, mid, last, p
            };
            c.set_ref_count(2);

            auto& left = *new (c.allocate_child()) sort_impl{first, mid, p};
            spawn(left);

            recycle_as_child_of(c);
            first = mid;
            return this;
          }
        };
    \end{minted}
    Kao što smo diskutovali pri opisivianju generalnih karakteristika paralelnog merge sort-a,
    i ovde definišemo \verb|cutoff| (ovde definisanu kao \verb|sort_cutoff|) konstantu koja tera
    algoritam da pređe u serijski režiim ukoliko je ulazni interval iteratora dovoljno mali. Mešavinom
    eksperimentalne metode i nezavisnog istraživanja, za ovu implementaciju definisaćemo \verb|sort_cutoff|
    kao 2048. Pri analizi performansi algoritama variraćemo vrednost \verb|cutoff| u cilju razumevanja
    ponašanja karakteristika performansi u odnosu na nju. Druga polovina \verb|execute| raspoređuje dva
    zadatka koji odgovaraju sortiranju polovina niza, uključujući optimizaciju stvaranja što manje instanci
    zadataka.\par
    Ideja je da određene zadatke koji nemaju šta da rade, a nisu uništeni jer čekaju druge zadatke
    recikliramo umesto da konstanto pravimo nove. Za ovo nam je koristan koncept kontinuacija.
    Kontinuacije su zadaci čija je namena da čekaju druge zadatke i eventualno odrade neki posao nakon čekanja.
    U \verb|parallel::sort_impl::execute| konstruisaćemo kontinuaciju \verb|c| (linije 17--20), koja će pratiti izvršavanje
    sortiranje polovina niza, pa konstruisati i pokrenuti novi zadatak za sortiranje levog podniza (linije 22--23) i reciklirati,
    trenutni zadatak tako da sortira desni podniz (linije 25--27).\par
    Posao \verb|parallel::sort_impl::sort_continuation| je jednostavan:
    \begin{minted}{cpp}
      template <typename It, typename P>
      class parallel::sort_impl : public task {
      private:
        // sort_impl member variables
        struct sort_continuation {
          // sort_continuation member variables
          auto execute() -> task* {
            parallel::merge(first, mid, last, p);
            return nullptr;
          }
        };
      public:
        // constructor
        // execute function
      };
    \end{minted}
    Ona samo čeka završetak zadataka koji su joj dodedljeni na čekanje (ovo je implicitno u tome da se
    \verb|sort_continuation| u \verb|sort_impl::execute| alocira kao kontinuacija i postavlja sve svoje podzadatke kao podzadatke kontinuacije), i nakon toga
    poziva funkciju koja paralelno spaja podnizove $[\verb|first|, \verb|mid|)$ i $[\verb|mid|, \verb|last|)$.\par
    Javni interfejs za poziv ovog algoritma svodi se na kreiranje i izvršavanje korenskog zadatka:
    \begin{minted}{cpp}
      template <typename It, typename P = less<>>
      void parallel::sort(It first, It last, P&& p = {}) {
        if (last - first <= sort_cutoff) {
          serial::sort(first, last, p);
          return;
        }
        
        auto& task = *new (task::allocate_root()) sort_impl{
          first, last, p
        };
        task::spawn_root_and_wait(task);
      }
    \end{minted}
    Dodaćemo na samom početku tela funkcije još jednu proveru za prelazak u serijski režim, da ne bismo
    trošili procesorske cikluse na alociranje TBB zadatka, koje je u principu netrivijalno skupo. Ovom transformacijom
    bi trebalo da smo izjednačili performanse paralelnog i serijskog algoritma za male nizove.\par
    \verb|parallel::merge_impl| klasa funkcioniše na sličan način:
    \begin{minted}{cpp}
      template <typename It, typename Out, typename P>
      class parallel::merge_impl : public task {
      private:
        // member variables and merge_continuation definition
      public:
        // constructor
        // ...
        auto execute() -> task* {
          auto l_size = static_cast<size_t>(l_last - l_first);
          auto r_size = static_cast<size_t>(r_last - r_first);

          if (l_size < r_size) {
            swap(l_first, r_first);
            swap(l_last, r_last);
            swap(l_size, r_size);
          }

          if (l_size == 0)
            return nullptr;

          if (l_size + r_size <= merge_cutoff) {
            serial::merge_impl(
              l_first, l_last, 
              r_first, r_last, 
              o_first, 
              p
            );
            return nullptr;
          }

          auto const midpt = l_first + l_size/2;
          auto const parpt = lower_bound(r_first, r_last, *midpt, p);
          auto const inspt = o_first + (midpt - l_first) + (parpt - r_first);
          *inspt = *midpt;

          auto& c = *new (allocate_continuation()) merge_continuation;
          c.set_ref_count(2);

          auto& right = *new (c.allocate_child()) merge_impl{
            midpt + 1, l_last,
            parpt, r_last,
            inspt + 1, 
            p 
          };
          spawn(right);

          recycle_as_child_of(c);
          l_last = midpt;
          r_last = parpt;

          return this;
        }
      };
    \end{minted}
    Na sličan način kao i u \verb|parallel::sort_impl|, i ovde koristimo \verb|cutoff| za prelazak
    u serijski režim rada (u ovom slučaju nazvana je \verb|merge_cutoff|), koja je takođe definisana kao 2048.
    U prvoj polovini \verb|execute| radimo posao koji smo izveli u diskusiji paralelnog merge algoritma.
    Zatim, na sličan način kao i u \verb|parallel::sort_impl| klasi, konstruišemo kontinuaciju \verb|c|,
    konstruišemo prvi podzadatak i recikliramo trenutni zadatak kao drugi podzadatak.\par
    U ovom slučaju, struktura
    koja predstavlja kontinuaciju je trivijalna, jer je spajanje podnizove poptuno u nadležnosti ne-kontinuacijskih
    zadataka (nasuprot sortiranju, gde nakon sortiranja podnizova moramo još i da ih spojimo).
    \begin{minted}{cpp}
      template <typename It, typename Out, typename P>
      class parallel::merge_impl : public task {
      private:
        // member variables
        struct merge_continuation : empty_task { };
      public:
        // constructor
        // execute function
      };
    \end{minted}
    \verb|tbb::empty_task| je subklasa \verb|tbb::task| čija \verb|execute| funkcija ne radi ništa, što
    je savršeno za naše potrebe, iz malopre navedenih razloga (setimo se još jednom da je čekanje na zadatke
    jedne kontinuacije implicitno definisano pri konstrukciji te kontinuacije i dodavanja podzadataka toj kontinuaciji).\par
    Da bismo adaptirali interfejs poziva paralelnog merge algoritma interfejsu \verb|serial::merge| i interfejsu
    koji očekuje njen poziv u \verb|sort_impl::execute|, moraćemo malo da se pomučimo:
    \begin{minted}{cpp}
      template <typename It, typename P = less<>>
      void merge(It first, It mid, It last, P&& p = {}) {
        auto const size = static_cast<size_t>(last - first);

        if (size <= merge_cutoff) {
          serial::merge(first, mid, last, p);
          return;
        }

        auto alloc = allocator<value_t<It>>{};
        auto *into = alloc.allocate(size);
        auto& task = *new (task::allocate_root()) merge_impl{
          first, mid,
          mid, last,
          into, 
          p
        };
        task::spawn_root_and_wait(task);

        parallel_for(
          blocked_range<size_t>{0, size},
          [&] (blocked_range<size_t> const& range) {
            for (auto i = range.begin(); i != range.end(); ++i) {
              auto pos = first + i;
              *pos = move(into[i]);
            }
          }
        );
        alloc.deallocate(into, size);
      }
    \end{minted}
    Slično kao i za paralelni merge sort, i ovde imamo dodatnu proveru za male intervale, dinamičku
    alokaciju preko \verb|std::allocator| i konstrukciju i pozivanje korenskog zadatka. Međutim,
    na kraju funkcije treba da iz privremenog niza u koji smo spojili podnizove pomerimo elemente u
    originalni niz. U serijskoj verziji, to smo radili jednostavnim pozivom funkcije \verb|std::move|
    nad relevantnim iteratorima, ali ovde imamo priliku da to paralelizujemo, što je savršena prilika
    da iskoristimo \verb|tbb::parallel_for|. Ako zanemarimo detalje postavke petlje, \verb|tbb::parallel_for|
    svodi se na običnu \verb|for| petlju (ili poziv \verb|std::for_each|), gde se na deklarativan način TBB biblioteci prepušta
    odgovornost raspoređivanja niti za izvršavanje iteracija petlje.
\newpage
    \section{Analiza performansi}
    \newpage

    \section{Zaključak}
    \newpage

    \section{Tabele}
    \newpage

    \section{Literatura}
    \newpage
\end{document}